\subsection{{Ambientes virtuais}}

A virtualização vem sendo muito utilizada na área de desenvolvimento de \textit{software} devido à sua flexibilidade e fácil manipulação para realizar a criação de ambientes virtuais. Isso ocorre porque, com a virtualização, o usuário pode criar vários ambientes virtuais e, dentro deles, realizar a instalação das dependências necessárias para a execução do projeto. Portanto, o desenvolvedor pode ativar e desativar o ambiente virtual assim que necessário, sendo que as dependências do projeto estarão instaladas somente nesse ambiente, e não globalmente na máquina. Ou seja, as dependências do projeto serão desativadas assim que a virtualização for encerrada. A utilização de virtualização permite ainda que recursos computacionais possam ser alocados para múltiplas aplicações simultaneamente, sendo que cada uma dessas aplicações possui seu ambiente isolado das demais.

A situação citada acima é bastante válida se analisarmos o crescimento de ferramentas de programação disponíveis atualmente. Instalar vários \textit{frameworks}, pacotes de dependências de linguagens, bibliotecas e afins globalmente na máquina pode acarretar a lentidão, conflitos entre versões de linguagens de programação e \textit{frameworks} dos projetos, dentre outros. Quando isso acontece, por exemplo, em um projeto feito por uma equipe, a atualização de todas as dependências deve ser feita em todas as máquinas que estão envolvidas no projeto, para que a versão seja padrão em toda a equipe.