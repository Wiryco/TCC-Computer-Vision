\section{\textbf{Descrição do sistema}}
\label{descricao-do-sistema}

% Antes de descrever qualquer coisa relacionada a ferramenta, é preciso entender que o \textit{software} passa por um aprendizado de máquina para aprender os padrões de características que são necessária para atender as necessidades do projeto, que é reconhecer um jogador em campo. O \textit{haar cascade} fica responsável por realizar a classificação da imagem para obter seus padrões de características. Basicamente, o \textit{haar cascade} é alimentado manualmente por imagens aleatórias de jogadores para montar um padrão de características e em seguida, esses padrões são treinados através do \textit{machine learning}. Isso deve ser feito para que seja possível obter um modelo de busca. Pode-se definir o modelo de busca como padrão de características de jogadores de futebol americano que será utilizado para realizar a busca.

O \textit{software} apresentado no decorrer deste projeto tem por objetivo reconhecer um jogador dentro de campo em tempo real, utilizando as técnicas de processamento de imagem para realizar a extração de características importantes presentes em uma imagem.

Antes de descrever qualquer coisa relacionada a ferramenta, é preciso entender que o \textit{software} passa por um aprendizado de máquina para entender os padrões de características que são necessária para atender as necessidades do projeto, que é reconhecer um jogador em campo. Portanto, o sistema utilizará classificação de imagem para extrair as características relacionadas aos jogadores de futebol americano.

O \textit{haar cascade} fica responsável por realizar a classificação da imagem para obter seus padrões de características. Basicamente, o mesmo é alimentado manualmente por imagens aleatórias de jogadores de futebol americano para montar esse padrão de características. Sendo assim, pode-se definir o modelo de busca como o conjunto de padrões de características de jogadores de futebol americano que será utilizado para realizar a busca.

Em seguida, o \textit{software} passa pela etapa de \textit{machine learning} que ficará encarregada apenas de treinar o algoritmo utilizando o modelo de busca. Ou seja, nessa etapa do processo, o sistema recebe como parâmetro os padrões de características extraídos na etapa anterior e o \textit{machine learning} fica analisando os seus pontos de interesse.

A etapa de aprendizado de máquina realiza um filtro que pode ser definido manualmente no algoritmo. No caso deste projeto, os filtros que foram criados e definidos para treinamento foram os de fisionomia de um jogador de futebol americano, as características do seu uniforme (capacete, número na camisa, tonalidade do uniforme) e os traços faciais. Após todas estas etapas, o \textit{software} estará pronto para realizar a busca de um jogador dentro de campo.

Em seguida, o sistema recebe, através do dispositivo de captura de imagem, um vídeo em tempo real de uma simulação de uma partida de futebol americano.

O algoritmo realiza a análise de todos os \textit{frames}\footnote{\textit{Frames} por segundo é a taxa de atualização de imagens estáticas que formam uma cena animada dentro de um vídeo. A ilusão que nosso cérebro interpreta como movimento é feita através de vários quadros consecutivos em um curto período de tempo \cite{FRAMES2011}. \label{frames-por-segundo}} por segundo do vídeo para encontrar algo similar ao modelo de busca que foi treinado.

Caso ocorra a identificação de um jogador de futebol americano dentro do vídeo analisado, o algoritmo fica encarregado de realizar uma representação do mesmo. Essa representação será feita com um contorno verde ao redor do jogador.