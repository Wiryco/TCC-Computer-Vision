\subsection{{Etapas de funcionamento}}

O funcionamento da ferramenta consiste em analisar uma imagem de um jogador objetivando adquirir todos os pontos de interesse da imagem, que servirá como parâmetro de busca para que o algoritmo seja mais eficiente quando o  mesmo for exposto a uma situação mais complexa como por exemplo, analisar um jogador com capacete.

A análise feita pelo algoritmo ocorre de forma minuciosa e utilizando técnicas de processamento de imagem. Sendo assim, após a extração de todos os pontos de interesse da imagem, o algoritmo monta um conjunto de sequências lógicas das taxas de \textit{pixels} existentes nesses pontos.

Após a montagem do conjunto, o algoritmo utiliza um dispositivo de captura de imagem para realizar uma busca por similaridade, ou seja, o algoritmo buscará algo semelhante ao padrão montado na etapa anterior. Após a busca feita pelo \textit{software}, a detecção vai definir se existe ou não um jogador no ambiente que está sendo analisado. Se for constatado que existe um jogador e ele for igual ao modelo, a ferramenta vai apresentar o seu nome cadastrado. 

Ao final do processo, será feito uma análise dos dados apresentados pela ferramenta. Através desses dados, pode-se obter o percentual de acertos e erros, analisando também os cenários de maior e menor eficiência.