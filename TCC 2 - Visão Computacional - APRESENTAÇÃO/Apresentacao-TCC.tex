\documentclass[aspectratio=169]{beamer}	  	

\usetheme{Pittsburgh}
\usecolortheme{default}
\usefonttheme[onlymath]{serif}			% para fontes matemáticas
% Enconte mais temas e cores em http://www.hartwork.org/beamer-theme-matrix/ 
% Veja também http://deic.uab.es/~iblanes/beamer_gallery/index.html

% Customizações de Cores: fg significa cor do texto e bg é cor do fundo

\definecolor{cor_titulo_azul}{RGB}{73, 76, 116}
\definecolor{cor_titulo_marrom}{RGB}{155, 138, 107}
\definecolor{cor_block_cinza}{RGB}{200, 200, 200}

\setbeamercolor{normal text}{fg=black}
\setbeamercolor{alerted text}{fg=cor_titulo_azul}
\setbeamercolor{author}{fg=black}
\setbeamercolor{title}{fg=cor_titulo_azul}
\setbeamercolor{institute}{fg=black}
\setbeamercolor{date}{fg=black}
\setbeamercolor{frametitle}{fg=cor_titulo_azul}
\setbeamercolor{framesubtitle}{fg=brown}
%Cor do título
\setbeamercolor{block title}{bg=cor_titulo_marrom, fg=white}
%Cor do texto (bg= fundo; fg=texto)
\setbeamercolor{block body}{bg=cor_block_cinza, fg=black}

% ---
% PACOTES
% ---
\usepackage[alf]{abntex2cite}		% Citações padrão ABNT
\usepackage[brazil]{babel}		% Idioma do documento
\usepackage{color}			% Controle das cores
\usepackage[T1]{fontenc}		% Selecao de codigos de fonte.
\usepackage{graphicx}			% Inclusão de gráficos
\usepackage[utf8]{inputenc}		% Codificacao do documento (conversão automática dos acentos)
\usepackage{txfonts}			% Fontes virtuais
% ---
\usepackage{graphicx}
\usepackage{changepage}
\usepackage{tikz}

\begin{document}

\begin{frame}

\title{\textbf{Visão computacional aplicada no reconhecimento de jogadores de futebol americano}}

\author{THAYRONE MARQUES SILVA
 \\	VINÍCIUS ANDRADE LOPES}
 
% \institute{FACULDADES DOCTUM DE IPATINGA
% \par BACHARELADO EM SISTEMAS DE INFORMAÇÃO}
\date{\today
%v-1.9.7
}

\begin{minipage}{1\linewidth}
  \centering
  \begin{tabular}{cc}
    \begin{tabular}{c}
      	\resizebox{.2\linewidth}{!}{\includegraphics{01-INFO_INICIAIS/doctum-logo-blue.png}}
    \end{tabular}
    &
    \begin{tabular}{c}
      \textbf{FACULDADES DOCTUM DE IPATINGA} \\ \textbf{BACHARELADO EM SISTEMAS DE INFORMAÇÃO}
    \end{tabular}
  \end{tabular}
\end{minipage}

\titlepage

\end{frame}

\begin{frame}{Sumário}
\tableofcontents
\end{frame}

\section{Introdução}

\begin{frame}{Introdução}

\begin{enumerate}
 \item {Visão Computacional.}
 \item {Futebol Americano.}
 \item {Avanço tecnológico.}
\end{enumerate}

\end{frame}

\section{Objetivos}

\subsection{Geral}
\begin{frame}{Objetivos}{Geral}
% \section{Objetivo Geral}
% \begin{frame}
% \frametitle{Objetivo Geral}

\begin{block}{}
 Apresentar uma ferramenta que seja capaz de identificar um jogador dentro de campo. Para isso, a identificação será feita através da ferramenta na qual será desenvolvida utilizando visão computacional e a biblioteca de processamento de imagens \textit{OpenCV}.
\end{block}

%\end{frame}
\end{frame}

\subsection{Específicos}
\begin{frame}{Objetivos}{Específicos}
\section{Objetivos Específicos}
\begin{frame}
\frametitle{Objetivos Específicos}

\begin{block}{}
\begin{itemize}
\item Capturar, através de um dispositivo de entrada de vídeo, as imagens de uma partida de futebol americano.

\item Processar as imagens capturadas e definir um modelo de busca de um jogador.
   
\item Identificar o jogador seguindo o modelo.
   
\item Analisar o percentual de acertos e erros da ferramenta.

\end{itemize}
\end{block}
\end{frame}
\end{frame}

%% \section{Objetivo Geral}
% \begin{frame}
% \frametitle{Objetivo Geral}

\begin{block}{}
 Apresentar uma ferramenta que seja capaz de identificar um jogador dentro de campo. Para isso, a identificação será feita através da ferramenta na qual será desenvolvida utilizando visão computacional e a biblioteca de processamento de imagens \textit{OpenCV}.
\end{block}

%\end{frame}

%\section{Objetivos Específicos}
\begin{frame}
\frametitle{Objetivos Específicos}

\begin{block}{}
\begin{itemize}
\item Capturar, através de um dispositivo de entrada de vídeo, as imagens de uma partida de futebol americano.

\item Processar as imagens capturadas e definir um modelo de busca de um jogador.
   
\item Identificar o jogador seguindo o modelo.
   
\item Analisar o percentual de acertos e erros da ferramenta.

\end{itemize}
\end{block}
\end{frame}

\section{Justificativa}
\begin{frame}
\frametitle{Justificativa}

\begin{itemize}
\item Crescimento exponencial e utilização em aplicações com soluções específicas.

\item Avanço tecnológico e dispositivos com poder computacional elevado.
   
\item Escalabilidade na utilização da solução para problemas mais complexos.

\item O mercado do futebol americano.

\end{itemize}

\end{frame}

%\section{Organização do Trabalho}
\begin{frame}
\frametitle{Organização do Trabalho}

Este trabalho é composto por cinco capítulos, estruturados da seguinte forma: o \autoref{cap-fundamentos-conceituais} é composto pelos fundamentos conceituais que foram necessários para o entendimento da área de visão computacional. No \autoref{metodologia} foi abordado a metodologia utilizada para a construção deste trabalho.  Subsequente, o \autoref{desenvolvimento} relata todo o desenvolvimento do trabalho, ressaltando todas as etapas de funcionamento, descrição do sistema e requisitos. Por fim, o \autoref{consideracoes_finais} apresenta as considerações finais obtidas com o projeto, seguida das análises dos resultados e possíveis estudos futuros.

\end{frame}

\section{Fundamentos Conceituais}

\subsection{Futebol Americano}
\begin{frame}{Fundamentos Conceituais}{Futebol Americano}
\begin{itemize}
    \item<1> Sua origem foi datada em 1876;
    \item<1> Características;
    \item<1> Recursos para alcançar melhores resultados.
\end{itemize}
\end{frame}

\subsection{Tecnologias aplicadas no esporte e o mercado do futebol americano}
\begin{frame}{Fundamentos Conceituais}{Tecnologias aplicadas no esporte e o mercado do futebol americano}
\begin{enumerate}
\item
    \begin{itemize}
    \item \citeonline{KATZ1989} explicam que com o uso de técnicas do meio desportivo incorporadas
    a tecnologia, é possível ampliar a performance e a inteligência dos atletas, fazendo com
    que as habilidades dos atletas seja cada vez mais aprimorada;
    \item Os lances com checagem do \textit{VAR} apresentam
    um atraso de 46\%, cerca de 1:50min \cite{ESTADAO2019}.
    \end{itemize}
\item
    \begin{itemize}
    \item O futebol americano é o esporte mais popular nos Estados Unidos e, segundo \citeonline{FORBES2018}, a NFL é a liga esportiva mais rica do mundo, girando cerca de U\$ 2,5 bilhões por cada time participante, operando com lucros de U\$ 101 milhões por franquia.
    
    \item “Fora os Estados Unidos, o Brasil é o segundo país mais interessado do mundo na NFL atualmente, perdendo apenas para o México \cite{FOLHASP2019}.”
    \end{itemize}
\end{enumerate}
\end{frame}

\subsection{Visão Computacional}
\begin{frame}{Fundamentos Conceituais}{Visão Computacional}
\begin{itemize}

\item Se aprimorou a ponto de chegar mais próximo da visão humana e até ser mais eficiente em algumas situações .

\item Abrange todas as técnicas e métodos de processamento de imagem.

\item Foi desenvolvida através da neurofisiologia da visão humana \cite{MARR76}.

\end{itemize}
\end{frame}

\subsection{Reconhecimento facial}
\begin{frame}{Fundamentos Conceituais}{Reconhecimento facial}
\begin{itemize}
    \item<1> Segundo \citeonline{SZELISKI2010}, a área de reconhecimento facial foi a que teve mais sucesso nos dias atuais.
    \item<1> Bibliotecas de reconhecimento.
    \item<1> Reconhecimento de jogadores.
\end{itemize}
\end{frame}

\subsection{Classificação de imagens}
\begin{frame}{Fundamentos Conceituais}{Classificação de imagens}
A classificação de imagem pode ser feita utilizando duas técnicas: supervisionada ou não-supervisionada \cite{LIBERMAN97}.

\begin{enumerate}
    \item<1> \textit{Haar Cascade} - Vetor de características;
    \item<1> \textit{Machine Learning}.
\end{enumerate}
\end{frame}

\subsection{Similaridade}
\begin{frame}{Fundamentos Conceituais}{Similaridade}
\begin{itemize}
    \item<1> Utiliza parâmetros de entrada para realizar a comparação de acordo com a necessidade do usuário. Nesse projeto em específica, será utilizado o modelo de busca feito na etapa de extração de características;
    \item<1> Realiza uma busca de objetos semelhantes ao modelo de busca;
    \item<1> Segundo \citeonline{MAIA2013}, algoritmos de similaridade trabalham com métricas que informam o quanto uma imagem é parecida com a outra.
\end{itemize}
\end{frame}

\section{Metodologia}
\begin{frame}
\frametitle{Metodologia}
\label{metodologia}

\end{frame}

\section{Desenvolvimento}
\begin{frame}
\frametitle{Desenvolvimento}
\label{desenvolvimento}

\end{frame}

\section{Considerações Finais}
\begin{frame}
\frametitle{Considerações Finais}
\label{consideracoes_finais}

\end{frame}

\section{Referências}

% --- O comando \allowframebreaks ---
% Se o conteúdo não se encaixa em um quadro, a opção allowframebreaks instrui 
% beamer para quebrá-lo automaticamente entre dois ou mais quadros,
% mantendo o frametitle do primeiro quadro (dado como argumento) e acrescentando 
% um número romano ou algo parecido na continuação.

\begin{frame}[allowframebreaks]{Referências}
\bibliography{references}
\end{frame}

\end{document}
