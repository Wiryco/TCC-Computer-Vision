\section{\textbf{{Engenharia de \textit{software}}}}
\label{engenharia-software}

Quando se pensa em desenvolvimento, manutenção, especificação e criação de um \textit{software}, pensa-se também em tecnologias e práticas de gerência de projetos para que a execução da aplicação aconteça de forma organizada, produtiva e com a máxima qualidade possível. Tudo isso esta contido em engenharia de \textit{software}.

Segundo \citeonline{PRESSMAN2016} em seu livro, um \textit{software} bem sucedido é aquele que atende a todos os requisitos do usuário, fica implementado durante um bom tempo, é de fácil manutenção e operabilidade. Por outro lado, um \textit{software} mal sucedido pode acarretar diversos fatos desagradáveis, levando os usuários a insatisfação e ao erro. 

Apesar de gerentes, líderes de projetos e profissionais envolvidos com a área técnica entenderem a necessidade de uma metodologia mais disciplinar no desenvolvimento de \textit{softwares}, existe ainda discursos de como e qual é a melhor metodologia a ser aplicada no projeto. Essa indecisão ocorre devido a grande demanda de produção que acontece atualmente, principalmente no setor de desenvolvimento de aplicações. Outro impacto negativo, é o fato de que profissionais e empresas começam a desenvolver \textit{softwares} de forma descontrolada mesmo com uma metodologia organizacional aplicada, justamente por não estarem preparados para uma abordagem disciplinar \cite{PRESSMAN2016}. 

Com base nisso, a engenharia de \textit{software} evoluiu rigorosamente, passando de uma simples técnica implementada por um publico relativamente pequeno para uma comunidade que objetiva o planejamento e a organização ates de iniciar qualquer tipo de desenvolvimento.

Sendo assim, em 2001, o engenheiro de \textit{software} Kent Beck juntamente com os principais
desenvolvedores de métodos ágeis, assinaram o “Manifesto para o Desenvolvimento Ágil de Software” \cite{SOMMERVILLE2011}, que tem por iniciativa a seguinte ideia:

\begin{citacao}
"Estamos descobrindo melhores maneiras de desenvolver \textit{software}, o fazendo e ajudando outros a fazê-lo. Através desse trabalho, valorizamos mais:

Indivíduos e interações do que processos e ferramentas;\\
\textit{Software} em funcionamento do que documentação abrangente;\\
Colaboração dos clientes acima de negociação contratual;\\
Respostas a mudanças acima de seguir um plano;\\
Ou seja, embora itens à direita sejam importantes, valorizamos mais os que estão à esquerda.\cite{SOMMERVILLE2011}"
\end{citacao}

\subsection{Engenharia de Requisitos}

Quando se pensa em projetar e construir um \textit{software}, os desafios e as imaginações tomam proporções escalonáveis para obter a melhor forma de iniciar e prosseguir com o desenvolvimento do projeto. O grande problema nesse processo é definir quais são os requisitos necessários para que o sistema atenda as necessidades do usuário.

Segundo \citeonline{PRESSMAN2005}, a engenharia de requisitos é basicamente uma etapa da engenharia de software, que deve ser iniciada durante as atividades de comunicação e continua no decorrer do desenvolvimento do \textit{software}. "Ela deve ser adaptada às necessidades do processo, do projeto, do produto e das pessoas que estão realizando o trabalho.”

A engenharia de requisitos tem por objetivo fornecer regras apropriadas para entender as necessidades do cliente afim de avaliar a viabilidade, negociar soluções razoáveis, validar as especificações e gerenciar as necessidades dos usuários na medida em que o sistema seja desenvolvido \cite{PRESSMAN2016}.

\input{4-Conteudo-Bibliografico/3-Engenharia-Software/desenvolvimento-agil-de-software.tex}