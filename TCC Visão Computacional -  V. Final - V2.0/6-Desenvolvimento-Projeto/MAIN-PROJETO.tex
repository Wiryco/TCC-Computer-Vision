\chapter{\textbf{DESENVOLVIMENTO}}
\label{desenvolvimento}

% Este capítulo tem por finalidade abordar todo o processo utilizado para o desenvolvimento desse projeto, relatando as principais ferramentas utilizadas e testes do algoritmo. O capítulo está dividido em seções, onde a \autoref{desenvolvimento-do-sistema} descreve o desenvolvimento do sistema, a seção \autoref{descricao-do-sistema} apresenta a descrição do sistema, etapas de funcionamento e seus riscos e restrições. A \autoref{requisitos} descreve os requisitos do sistema, a \autoref{explicacao-codigo} tem por objetivo descrever os códigos utilizados para a elaboração deste projeto e a \autoref{ambientes-de-teste} relata os ambientes de teste e os testes do software.

Este capítulo tem por finalidade abordar todo o processo utilizado para o desenvolvimento desse projeto, relatando as principais ferramentas utilizadas e testes do algoritmo. O capítulo está dividido em seções, onde a \autoref{descricao-do-sistema} apresenta a descrição do sistema, etapas de funcionamento e seus riscos e restrições. A \autoref{requisitos} descreve os requisitos do sistema, a \autoref{explicacao-codigo} tem por objetivo descrever os códigos utilizados para a elaboração deste projeto e a \autoref{testes_do_software} relata os testes e as análises do \textit{software}.

%\section{\textbf{Desenvolvimento do sistema}}
\label{desenvolvimento-do-sistema}

COMO FOI FEITO

\section{\textbf{Descrição do sistema}}
\label{descricao-do-sistema}

Antes de descrever qualquer coisa relacionada a ferramenta, é preciso entender que o \textit{software} passa por um aprendizado de máquina (\autoref{itm:machinelearning}) para aprender os padrões de características que são necessária para atender as necessidades do projeto, que é reconhecer um jogador em campo. O \textit{haar cascade} (\autoref{itm:haarcascade}) fica responsável por realizar a classificação da imagem para obter seus padrões de características. Basicamente, o \textit{haar cascade} é alimentado manualmente por imagens aleatórias de jogadores para montar um padrão de características e em seguida, esses padrões são treinados através do \textit{machine learning}. Isso deve ser feito para que seja possível obter um modelo de busca. Pode-se definir o modelo de busca como padrão de características de jogadores de futebol americano que será utilizado para realizar a busca.

\subsection{{Etapas de funcionamento}}

O funcionamento da ferramenta consiste em analisar uma imagem de um jogador objetivando adquirir todos os pontos de interesse da imagem, que servirá como parâmetro de busca para que o algoritmo seja mais eficiente quando o  mesmo for exposto a uma situação mais complexa como por exemplo, analisar um jogador com capacete.

A análise feita pelo algoritmo ocorre de forma minuciosa e utilizando técnicas de processamento de imagem. Sendo assim, após a extração de todos os pontos de interesse da imagem, o algoritmo monta um conjunto de sequências lógicas das taxas de \textit{pixels} existentes nesses pontos.

Após a montagem do conjunto, o algoritmo utiliza um dispositivo de captura de imagem para realizar uma busca por similaridade, ou seja, o algoritmo buscará algo semelhante ao padrão montado na etapa anterior. Após a busca feita pelo \textit{software}, a detecção vai definir se existe ou não um jogador no ambiente que está sendo analisado. Se for constatado que existe um jogador e ele for igual ao modelo, a ferramenta vai apresentar o seu nome cadastrado. 

Ao final do processo, será feito uma análise dos dados apresentados pela ferramenta. Através desses dados, pode-se obter o percentual de acertos e erros, analisando também os cenários de maior e menor eficiência.

\subsection{Restrições, riscos e exclusões do projeto}

As restrições do projeto podem ser definidas através de todos os fatores que limitam as funcionalidades do mesmo. Basicamente, são condições impostas para a elaboração do projeto que devem ser obrigatoriamente cumpridas pela equipe no decorrer do desenvolvimento do sistema. Com base nisso, as restrições deste projeto são:

\begin{itemize}
\raggedright \item Reconhecer pelo menos um jogador dentro de campo;
\raggedright \item Fazer a representação do jogador reconhecido.
% \item Somente o jogador que foi utilizado como modelo será representado. Caso algum outro jogador apareça na cena analisada, ele pode ser reconhecido como um jogador mais não terá nenhuma representação de reconhecimento, ou seja, o \textit{software} vai informar que o indivíduo é um jogador mas ele não o reconhecerá pelo nome.
\end{itemize}

Já a parte de riscos está relacionada a um evento ou situação incerta que pode afetar positivamente ou negativamente na execução do projeto. Descrevê-los é uma necessidade, para que saibamos identificar o momento certo que pode acontecer algum problema com relação ao software. Os riscos deste projeto são:

\begin{itemize}
\raggedright \item Se não houver uma boa iluminação, pode acontecer falhas no reconhecimento e até mesmo inutilizá-lo;
\raggedright \item A qualidade do sensor utilizado para capturar a imagem pode interferir no reconhecimento.
\end{itemize}

A exclusão consiste em todos os requisitos que estão explicitamente fora do projeto. Basicamente, é tudo aquilo que a equipe de desenvolvimento deixa claro que não será realizado ao longo do desenvolvimento do projeto. Sendo assim, segue a relação de requisitos que estão excluídos deste projeto:

\begin{itemize}
\raggedright \item Explicar matematicamente cada função executada;

\raggedright \item Reconhecer mais de um jogador dentro de campo;

\raggedright \item Solucionar algum possível problema identificado no processo de reconhecimento.
\end{itemize}

\section{\textbf{Requisitos}}
\label{requisitos}

Requisitos de um sistema podem ser definidos como toda a solicitação, necessidade, funcionalidade, característica, especificação ou gerenciamento que são necessários para atender um usuário ou alguma organização.

\subsection{{Requisitos funcionais}}

Requisitos funcionais são responsáveis pelas funcionalidades, necessidades e características que definem um software, ou seja, eles definem o que o \textit{software} vai fazer, o seu comportamento. Os requisitos funcionais deste projeto são:

\begin{itemize}
\raggedright \item Realizar a classificação das imagens de jogadores de futebol americano.

\raggedright \item Gerar o \textit{haar cascade} com os pontos de interesses das imagens classificadas.

\raggedright \item Definir o modelo de busca a partir dos padrões de características dos jogadores de futebol americano.

\raggedright \item Realizar o treinamento do algoritmo com o modelo de busca definido.

\raggedright \item Capturar o vídeo através de um dispositivo (\textit{hardware}).

\raggedright \item Analisar cada \textit{frame} por segundo do vídeo.

\raggedright \item Identificar um jogador de futebol americano dentro de campo.

\raggedright \item Representar o jogador analisado.
\end{itemize}

\subsection{{Requisitos não-funcionais}}

Basicamente, requisitos não funcionais descrevem como o sistema fará alguma coisa. Sendo assim, os requisitos não funcionais estão relacionados ao desempenho do sistema, usabilidade, confiabilidade, restrições de projeto, manutenção e atributos da qualidade. São os seguintes:

\begin{itemize}
\raggedright \item O sistema deve ser implementado em \textit{phyton};

\raggedright \item O sistema necessita de dependências do \textit{phyton} para a sua execução;

\raggedright \item A biblioteca \textit{OpenCV} precisa estar instalada para a sua execução;

\raggedright \item O processamento da imagem será feito em \textit{hardware} próprio;

\raggedright \item O sistema requer grande processamento de \textit{hardware};

\raggedright \item Para a definição de um novo modelo, o sistema precisa passar por uma manutenção dos parâmetros de busca e, consequentemente, um novo treinamento da ferramenta;

\raggedright \item Para que o sistema possa ser executado, é necessário realizar comandos via terminal;

\raggedright \item O \textit{software} precisa de um dispositivo de captura de imagem como, por exemplo, uma \textit{Web Cam};

\raggedright \item O \textit{software} deve poder ser executado em \textit{Windows} e \textit{Linux}.

\end{itemize}

\section{\textbf{Explicando o código}}
\label{explicacao-codigo}
\definecolor{cinza}{rgb}{0.85,0.85,0.85}
%\numberwithin{listing}{section}

% Antes de explicar o código por trás do desenvolvimento deste projeto, é preciso informar que para executá-lo é necessário instalar a linguagem de programação \textit{Python} e suas dependências.

Para iniciar a codificação do algoritmo em \textit{Python}, é necessário realizar as importações das bibliotecas necessárias para realizar o procedimento de reconhecimento. A biblioteca \textit{cv2} é a \textit{OpenCV}; a \textit{NumPy} fica responsável por realizar cálculos de vetores multidimensionais; a \textit{argparse} verifica e atribui os argumentos que são esperados; já a \textit{imutils} é responsável por converter a imagem em uma matriz (\autoref{codigo1}).

\begin{listing}[ht]
\caption{\label{codigo1}Importação de bibliotecas.}
\begin{minted}[linenos=true, breaklines=true, mathescape, bgcolor=cinza, breaklines, frame=single]{python}

import cv2
import numpy as np
import argparse
import imutils

\end{minted}
\end{listing}

Também é necessário importar os módulos da biblioteca \textit{imutils} (\autoref{codigo2}).

%\clearpage

\begin{listing}[ht]
\caption{\label{codigo2}Importação de modulos.}
\begin{minted}[linenos=true, breaklines=true, mathescape, bgcolor=cinza, breaklines, frame=single]{python}

from __future__ import print_function
from imutils.object_detection import non_max_suppression
from imutils import paths

\end{minted}
\end{listing}

Em seguida, o algoritmo deve ser programado para construir os argumentos necessários para iniciar o reconhecimento e analisar esses argumentos (\autoref{codigo3}).

\clearpage

\begin{listing}[ht]
\caption{\label{codigo3}Argumentos de reconhecimento.}
\begin{minted}[linenos=true, breaklines=true, mathescape, bgcolor=cinza, breaklines, frame=single]{python}

ap = argparse.ArgumentParser()
ap.add_argument("-i", "--images", required=True,
    help="path to images directory")
args = vars(ap.parse_args())

\end{minted}
\end{listing}

Agora é necessário iniciar o descritor \textit{HOG - Histogram of Oriented Gradients} (Histograma de Gradientes Orientados) (\autoref{codigo4}). Basicamente, o \textit{HOG} utiliza cálculos matemáticos complexos que serve para o reconhecimento de padrões e processamento de imagem, obtendo assim um conjunto robusto de características das imagens analisadas.

\begin{listing}[ht]
\caption{\label{codigo4}Descritor \textit{HOG - Histogram of Oriented Gradients}.}
\begin{minted}[linenos=true, breaklines=true, mathescape, bgcolor=cinza, breaklines, frame=single]{python}

hog = cv2.HOGDescriptor()
hog.setSVMDetector(cv2.HOGDescriptor_getDefaultPeopleDetector())

\end{minted}
\end{listing}

Em sequência, foi criado uma pasta local contendo todas as imagens necessárias para realizar a extração dos padrões de características dos jogadores de futebol americano. Sendo assim, o trecho de código a seguir é útil na leitura de todas estas imagens que estão dentro da pasta \textit{images} localmente (\autoref{codigo5}).

\begin{listing}[ht]
\caption{\label{codigo5}Diretório de imagens.}
\begin{minted}[linenos=true, breaklines=true, mathescape, bgcolor=cinza, breaklines, frame=single]{python}

imagePaths = list(paths.list_images(args["images"]))

\end{minted}
\end{listing}

A partir desta etapa, o algoritmo vai executar uma estrutura de repetição para analisar todas as fotos alocadas na variável \textit{imagePaths} utilizada no trecho de código acima.

Sendo assim, o primeiro laço da estrutura de repetição fica responsável por carregar e redimensionar a imagem para reduzir o tempo de detecção. Em seguida, ele melhora a precisão da detecção da imagem para melhorar a extração de características (\autoref{codigo6}).

\begin{listing}[ht]
\caption{\label{codigo6}Estrutura de repetição responsável por carregar e redimensionar as imagens.}
\begin{minted}[linenos=true, breaklines=true, mathescape, bgcolor=cinza, breaklines, frame=single]{python}

for imagePath in imagePaths:
	image = cv2.imread(imagePath)
	image = imutils.resize(image, width=min(400, image.shape[1]))
	orig = image.copy()

\end{minted}
\end{listing}

Em seguida, o algoritmo tenta detectar se existe uma pessoa na imagem analisada (\autoref{codigo7}).

\begin{listing}[ht]
\caption{\label{codigo7}Detecção do objeto na imagem.}
\begin{minted}[linenos=true, breaklines=true, mathescape, bgcolor=cinza, breaklines, frame=single]{python}

    (rects, weights) = hog.detectMultiScale(image, winStride=(4, 4),
        padding=(8, 8), scale=1.05)

\end{minted}
\end{listing}

Após isso, a segunda estrutura de repetição (\autoref{codigo8}) que está alocada dentro da primeira (\autoref{codigo6}), fica responsável por delimitar os locais onde contém os pontos de interesse dentro da imagem analisada, ou seja, o algoritmo seleciona as áreas de interesse que serão utilizadas como parâmetro de busca.

\begin{listing}[ht]
\caption{\label{codigo8}Estrutura de repetição responsável por delimitar os pontos de interesse na imagem.}
\begin{minted}[linenos=true, breaklines=true, mathescape, bgcolor=cinza, breaklines, frame=single]{python}

    for (x, y, w, h) in rects:
	cv2.rectangle(orig, (x, y), (x + w, y + h), (0, 0, 255), 2)

\end{minted}
\end{listing}

Em seguida, o \textit{software} realiza a técnica de supressão não máxima na imagem, ou seja, ele realiza a técnica de detecção de bordas para identificar os \textit{pixel} de maior intensidade e, em seguida, ele realiza a técnica de supressão para eliminar os \textit{pixels} que não possuem valores próximos aos da borda (\autoref{codigo9}).

\begin{listing}[ht]
\caption{\label{codigo9}Técnica de supressão não máxima (detecção de bordas).}
\begin{minted}[linenos=true, breaklines=true, mathescape, bgcolor=cinza, breaklines, frame=single]{python}

    rects = np.array([[x, y, x + w, y + h] for (x, y, w, h) in rects])
	pick = non_max_suppression(rects, probs=None, overlapThresh=0.65)

\end{minted}
\end{listing}

A terceira estrutura de repetição fica responsável por montar uma delimitação do indivíduo detectado na imagem analisada (\autoref{codigo10}).

%\clearpage

\begin{listing}[ht]
\caption{\label{codigo10}Delimitação do objeto identificado.}
\begin{minted}[linenos=true, breaklines=true, mathescape, bgcolor=cinza, breaklines, frame=single]{python}

    for (xA, yA, xB, yB) in pick:
		cv2.rectangle(image, (xA, yA), (xB, yB), (0, 255, 0), 2)

\end{minted}
\end{listing}

Apos os procedimentos, o algoritmo fica responsável por desenhar as correspondências de informações contidas entre as imagens analisadas (\autoref{codigo11}).

\begin{listing}[ht]
\caption{\label{codigo11}Desenho das correspondências.}
\begin{minted}[linenos=true, breaklines=true, mathescape, bgcolor=cinza, breaklines, frame=single]{python}

    filename = imagePath[imagePath.rfind("/") + 1:]
	print("[INFO] {}: {} original boxes, {} after suppression".format
		(filename, len(rects), len(pick)))

\end{minted}
\end{listing}

Por fim, o sistema mostra duas imagens na tela, onde a primeira representa uma tentativa de identificar o conteúdo da imagem utilizando o \textit{haar cascade} e a técnica de aprendizado. Já na segunda imagem, a representação dos conteúdos são feitas após um aperfeiçoamento das características extraídas da imagem (\autoref{codigo12}). A \autoref{fig_comparativo_img} representa o resultado do algoritmo.

\begin{listing}[ht]
\caption{\label{codigo12}Comandos para representar os resultados.}
\begin{minted}[linenos=true, breaklines=true, mathescape, bgcolor=cinza, breaklines, frame=single]{python}

cv2.imshow("Before NMS", orig)
cv2.imshow("After NMS", image)
cv2.waitKey(0)

\end{minted}
\end{listing}

\begin{figure}[h]
	\caption{\label{fig_comparativo_img}\textbf{TESTE 01:} (A) – \textit{Haar-cascade} e aprendizado de máquina. (B) – Utilizando a técnica de aperfeiçoamento.}
	\begin{center}
		\resizebox{.8\linewidth}{!}{\includegraphics{6-Desenvolvimento-Projeto/imagens-desenvolvimento/comparativo_imagem.png}}
	\end{center}
	\centering \legend{Fonte: Elaborada pelos autores.}
\end{figure}

\section{\textbf{Testes e análises do \textit{software}}}
\label{testes_do_software}
O \textit{software} criado no decorrer deste projeto é uma simples ideia de como usar técnicas de visão computacional aplicadas no reconhecimento de um jogador de futebol americano. Para exemplificar a usabilidade e os pontos fortes das técnicas utilizadas durante o desenvolvimento do mesmo, foram criadas várias situações para testar o desempenho do algoritmo. Sendo assim, as figuras a seguir representam os testes feitos com o sistema desenvolvido.

Inicialmente, o algoritmo foi testado em um cenário com imagens estáticas, como pode ser visto na \autoref{fig_comparativo_img} e \autoref{fig_rec_numero}, no qual ele analisa esta imagem, extraindo todos os seus padrões de características. Em seguida, ele realiza uma busca por similaridade na mesma imagem onde podemos notar, com maior evidência na  \autoref{fig_comparativo_img}, que mesmo ele cometendo alguns erros na parte de extração de características, a etapa de aprendizado de máquina entende os padrões e realiza um balanceamento de quais característica são interessantes para serem levadas em consideração na hora de montar um modelo de busca. As características de menor interesse são dispensadas para reduzir a probabilidade do algoritmo cometer erros.

\begin{figure}[ht]
	\caption{\label{fig_rec_numero}A imagem (A) representa os pontos de interesse encontrados nos números das camisas dos jogadores. A imagem (B) a sua identificaç ão.}
	\begin{center}
		\resizebox{1.0\linewidth}{!}{\includegraphics{6-Desenvolvimento-Projeto/imagens-desenvolvimento/representacao_numero.png}}
	\end{center}
	\centering \legend{Fonte: Elaborada pelos autores.}
\end{figure}

Em seguida, o algoritmo foi colocado em um cenário de análise de vídeo de uma partida de futebol americano. Nas etapas a seguir, o algoritmo utiliza o modelo de busca já treinado para identificar um jogador de futebol americano dentro de campo. Ou seja, não foi realizada nenhuma extração de característica de imagem, somente a técnica de busca por similaridade.


\begin{figure}[ht]
	\caption{\label{fig_rep_jogador_em_campo}Identificação de um jogador em uma partida de futebol americano.}
	\begin{center}
		\resizebox{.7\linewidth}{!}{\includegraphics{6-Desenvolvimento-Projeto/imagens_teste/identificacao_jogador_em_campo_2.png}}
	\end{center}
	\centering \legend{Fonte: Elaborada pelos autores.}
\end{figure}

A \autoref{fig_rep_jogador_em_campo} foi retirada de um vídeo de uma partida de futebol americano. O algoritmo analisou \textit{frame} a \textit{frame} do vídeo para realizar a busca por similaridade seguindo o modelo de busca já treinado. Com base nessas informações, pode-se notar que o \textit{software} realizou uma leitura dos jogadores de futebol americano e identificou o jogador que mais se aproxima das características contidas no modelo de busca.

Outro fator que pode ser visto na \autoref{fig_rep_jogador_em_campo} é que o algoritmo identificou somente um jogador na cena analisada. Isso ocorre porque o comportamento seguido pelo \textit{software} e de analisar o jogador por inteiro e compará-lo com os padrões de características do modelo de busca, analisando também a sua fisionomia. Sendo assim, os outros jogadores podem ter atendido algum ponto de característica contido dentro do modelo de busca, porém, aquele jogador identificado é o que mais se assimila ao modelo e por isso ele foi identificado na cena.

Já a \autoref{fig_rep_jogador_mais_evidente} mostra a identificação do jogador mais evidente no ponto de vista do algoritmo. Como pode ser visto, no meio de uma jogada de futebol americano, o jogador que mais se encaixou nos padrões de características do modelo de busca foi identificado. Novamente, como citado na análise da \autoref{fig_rep_jogador_em_campo}, o algoritmo identificou somente um jogador dentre os demais, pois ele é o que mais se encaixa no modelo de busca.

\begin{figure}[ht]
	\caption{\label{fig_rep_jogador_mais_evidente}Identificação do jogador mais evidente.}
	\begin{center}
		\resizebox{.7\linewidth}{!}{\includegraphics{6-Desenvolvimento-Projeto/imagens_teste/jogador_mais_evidente.png}}
	\end{center}
	\centering \legend{Fonte: Elaborada pelos autores.}
\end{figure}

Da mesma forma, a \autoref{fig_rep_jogador_em_movimento} representa a identificação de um jogador em movimento. Ao testar o \textit{software} nesse cenário, ele consegue, por um breve período, seguir o jogador analisado na sua corrida. Esse tempo não é o suficiente para analisar um jogador dentro de campo durante uma partida inteira, mas mostra que, com o aprimoramento do algoritmo e também o aprimoramento do modelo de busca, esse tipo de situação pode ser feita com mais eficiência.

%\clearpage

\begin{figure}[ht]
	\caption{\label{fig_rep_jogador_em_movimento}Identificação de um jogador em movimento dentro de campo.}
	\begin{center}
		\resizebox{.7\linewidth}{!}{\includegraphics{6-Desenvolvimento-Projeto/imagens_teste/jogador_em_movimento.png}}
	\end{center}
	\centering \legend{Fonte: Elaborada pelos autores.}
\end{figure}

A \autoref{fig_rep_jogador_em_jogada} representa um jogador em uma velocidade mais alta do que o da \autoref{fig_rep_jogador_em_movimento}. Isso ocorreu devido a um lance que exigiu mais condicionamento físico do jogador de futebol americano. Mesmo nesse cenário, o algoritmo também foi capaz de identificar e perseguir o jogador dentro de campo por um intervalo de tempo. No entanto, o algoritmo teve um pouco mais de dificuldade para capturar todos os movimentos e realizar a identificação do  jogador devido o seu excesso de velocidade, não sendo possível persegui-lo durante um bom tempo ou até mesmo quando entra em contato com o seu adversário.

%\clearpage

\begin{figure}[ht]
	\caption{\label{fig_rep_jogador_em_jogada}Identificação de um jogador em movimento para disputar uma jogada.}
	\begin{center}
		\resizebox{.7\linewidth}{!}{\includegraphics{6-Desenvolvimento-Projeto/imagens_teste/identificacao_jogadores_fa3.png}}
	\end{center}
	\centering \legend{Fonte: Elaborada pelos autores.}
\end{figure}

A \autoref{fig_rep_jogador_em_movimento_1} e \autoref{fig_rep_jogador_em_movimento_2} mostra como foi a tentativa do algoritmo de seguir um jogador em movimento para disputar uma jogada dentro de campo.

\begin{figure}[ht]
	\caption{\label{fig_rep_jogador_em_movimento_1}Tentativa de seguir os movimentos de um jogador dentro de campo.}
	\begin{center}
		\resizebox{.8\linewidth}{!}{\includegraphics{6-Desenvolvimento-Projeto/imagens_teste/jogador_dentro_de_campo_1.png}}
	\end{center}
	\centering \legend{Fonte: Elaborada pelos autores.}
\end{figure}

%\clearpage

\begin{figure}[ht]
	\caption{\label{fig_rep_jogador_em_movimento_2}Identificação do jogador de futebol americano após o movimento de corrida.}
	\begin{center}
		\resizebox{.8\linewidth}{!}{\includegraphics{6-Desenvolvimento-Projeto/imagens_teste/jogador_dentro_de_campo_9.png}}
	\end{center}
	\centering \legend{Fonte: Elaborada pelos autores.}
\end{figure}

Após a análise do \textit{frame} do vídeo representado pela \autoref{fig_rep_jogador_em_movimento_1}, no qual ele captura um movimento do jogador de futebol americano, foi realizada a mesma análise utilizando o algoritmo que analisa imagens estáticas. A \autoref{fig_rep_erro} mostra que, dependendo da situação de movimento do jogador, o algoritmo não o reconhece, e sim reconhece o seu movimento. Isso ocorre porque a maioria das imagens que foram utilizadas para treinar o algoritmo continha jogadores de futebol americano em suas devidas situações dentro de campo, ou seja, os jogadores estavam correndo, em ataque, em disputa de bola e em várias outras situações de jogo. Sendo assim, os movimentos dos jogadores de futebol americano dentro de cambo também é um fator de análise do algoritmo e, consequentemente, é algo que pode interferir na identificação do mesmo.

\begin{figure}[ht]
	\caption{\label{fig_rep_erro}Identificação do movimento de um jogador de futebol americano.}
	\begin{center}
		\resizebox{.7\linewidth}{!}{\includegraphics{6-Desenvolvimento-Projeto/imagens_teste/erro_algoritmo.png}}
	\end{center}
	\centering \legend{Fonte: Elaborada pelos autores.}
\end{figure}


A \autoref{fig_rep_dois_jogadores} representa basicamente o desempenho do algoritmo ao analisar o \textit{frame} do vídeo e identificar mais de um jogador de futebol americano dentro de campo. Como pode ser visto, o algoritmo conseguiu reconhecer dois jogadores dentro do \textit{frame} do vídeo, no entanto ele ainda não conseguiu identificar o restante. Nessa ocasião em específico, o algoritmo não seguiu com a identificação dos dois jogadores por um período longo e nem curto, pois ele “se perde” nas características e permanece identificando apenas um jogador.

\begin{figure}[ht]
	\caption{\label{fig_rep_dois_jogadores}Identificação de dois jogadores de futebol americano dentro de campo.}
	\begin{center}
		\resizebox{.7\linewidth}{!}{\includegraphics{6-Desenvolvimento-Projeto/imagens_teste/dois_jogadores_detectados.png}}
	\end{center}
	\centering \legend{Fonte: Elaborada pelos autores.}
\end{figure}

Já na \autoref{fig_rep_acessorios} podemos notar que o algoritmo não detectou nenhum jogador de futebol americano no \textit{frame} do vídeo. No entanto, como pode ser visto, o algoritmo conseguiu identificar alguns detalhes no uniforme dos jogadores. Isso ocorreu porque a variação de temperatura de cor naquela região atendeu aos padrões de características extraídas das imagens de treino que foram utilizadas para gerar o modelo de busca. Dessa forma, o algoritmo entendeu que, nesse \textit{frame} em específico, os pontos de maior interesse que possuem semelhança com o modelo de busca são os identificados na imagem.

\begin{figure}[ht]
	\caption{\label{fig_rep_acessorios}Identificação dos detalhes do uniforme de um jogador de futebol americano.}
	\begin{center}
		\resizebox{.8\linewidth}{!}{\includegraphics{6-Desenvolvimento-Projeto/imagens_teste/identificacao_jogadores_fa5.png}}
	\end{center}
	\centering \legend{Fonte: Elaborada pelos autores.}
\end{figure}

Outro fator muito interessante a ser analisado é a identificação feita pelo algoritmo na \autoref{fig_rep_arbitro}. Como pode ser visto, o sistema identificou o árbitro dentro de campo, e não  um dos jogadores de futebol americano. Essa situação foi bastante inusitada, pois não foi alterado nenhum parâmetro de busca e o modelo de busca também não foi alterado. Sendo assim, com base na \autoref{fig_rep_acessorios} e \autoref{fig_rep_arbitro}, pode-se concluir que a diferença de tonalidade dos uniformes também é um parâmetro que o algoritmo leva em consideração para realizar o processamento de imagem.

\clearpage

\begin{figure}[ht]
	\caption{\label{fig_rep_arbitro}Reconhecimento do árbitro dentro de campo.}
	\begin{center}
		\resizebox{.8\linewidth}{!}{\includegraphics{6-Desenvolvimento-Projeto/imagens_teste/identificacao_arbitro.png}}
	\end{center}
	\centering \legend{Fonte: Elaborada pelos autores.}
\end{figure}

A \autoref{fig_processamento_maquina}  representa o consumo de \textit{hardware} quando executamos o algoritmo para realizar a análise de um vídeo. Como pode ser visto, o \textit{software} requer muito poder de processamento da máquina, consumindo grande parte dos seus recursos.  Vale lembrar que este teste foi bem simples e mostra superficialmente os recursos necessários para a execução do algoritmo. Na \autoref{novos_estudos} foi sugerido um estudo mais avançado sobre esse consumo de \textit{hardware}.

%\clearpage

\begin{figure}[ht]
	\caption{\label{fig_processamento_maquina}Representação do processamento da máquina.}
	\begin{center}
		\resizebox{.7\linewidth}{!}{\includegraphics{6-Desenvolvimento-Projeto/imagens_teste/processamento_maquina.png}}
	\end{center}
	\centering \legend{Fonte: Elaborada pelos autores.}
\end{figure}

\clearpage

Para resumir os resultados obtidos nos testes e análise do \textit{software}, foi feito uma tabela (\autoref{resultado_de_testes}) que tem por finalidade representar quais foram os ambientes de testes que o algoritmo mais se destacou.

\begin{table}[h]
\centering
\caption{Resumo dos resultados dos testes}
\label{resultado_de_testes}
\begin{tabular}{l|l} 
\hline
\hline
\multicolumn{1}{l|}{\textbf{Ambiente de Teste}} & \multicolumn{1}{l}{\textbf{Resultado}}  \\ 
\hline
\centering
Teste 1 - \autoref{fig_comparativo_img} & Acertou parcialmente\\
Teste 2 - \autoref{fig_rec_numero} & Acertou\\
Teste 3 - \autoref{fig_rep_jogador_em_campo} & Acertou\\
Teste 4 - \autoref{fig_rep_jogador_mais_evidente} & Acertou\\
Teste 5 - \autoref{fig_rep_jogador_em_movimento} & Acertou\\
Teste 6 - \autoref{fig_rep_jogador_em_jogada} & Acertou\\
Teste 7 - \autoref{fig_rep_jogador_em_movimento_1} & Errou\\
Teste 8 - \autoref{fig_rep_jogador_em_movimento_2} & Acertou\\
Teste 9 - \autoref{fig_rep_erro} & Errou\\
Teste 10 - \autoref{fig_rep_dois_jogadores} & Superou as expectativas\\
Teste 11 - \autoref{fig_rep_acessorios} & Acertou parcialmente\\
Teste 12 - \autoref{fig_rep_arbitro} & Errou\\
Teste 13 - \autoref{fig_processamento_maquina} & Desempenho da máquina\\
\hline
\hline
\end{tabular}
\centering \legend{Fonte: Elaborada pelos autores.}
\end{table}

%\section{\textbf{Análise dos resultados}}

Para resumir os resultados obtidos nos testes de \textit{software}, foi feito uma tabela que tem por finalidade representar quais foram os ambientes de testes que o algoritmo foi submetido. Sendo assim, a \autoref{resultado_de_testes} mostra os resultados dos testes do algoritmo.

\begin{table}[h]
\centering
\caption{Resumo dos resultados dos testes}
\label{resultado_de_testes}
\begin{tabular}{l|l} 
\hline
\hline
\multicolumn{1}{l|}{Ambiente de Teste} & \multicolumn{1}{l}{Resultado}  \\ 
\hline
Partida de futebol americano & Positivo\\
Partida de futebol americano & Negativo\\
Jogador parado & Positivo\\
Jogador em movimento & Intermediário\\
Jogador de frente & Positivo\\
Jogador de costas & Positivo\\
\hline
\hline 
\end{tabular}
\end{table}