\subsection{Engenharia de Requisitos}

Quando se pensa em projetar e construir um \textit{software}, os desafios e as imaginações tomam proporções escalonáveis para obter a melhor forma de iniciar e prosseguir com o desenvolvimento do projeto. O grande problema nesse processo é definir quais são os requisitos necessários para que o sistema atenda as necessidades do usuário.

Segundo \citeonline{PRESSMAN2005}, a engenharia de requisitos é basicamente uma etapa da engenharia de software, que deve ser iniciada durante as atividades de comunicação e continua no decorrer do desenvolvimento do \textit{software}. "Ela deve ser adaptada às necessidades do processo, do projeto, do produto e das pessoas que estão realizando o trabalho.”

A engenharia de requisitos tem por objetivo fornecer regras apropriadas para entender as necessidades do cliente afim de avaliar a viabilidade, negociar soluções razoáveis, validar as especificações e gerenciar as necessidades dos usuários na medida em que o sistema seja desenvolvido \cite{PRESSMAN2016}.