\subsection{O mercado do futebol americano}
\label{mercado-do-futebol-americano}

O futebol americano é o esporte mais popular nos Estados Unidos e, segundo \citeonline{FORBES2018}, a \textit{NFL} é a liga esportiva mais rica do mundo, girando cerca de U\$ 2,5 bilhões por cada time participante, operando com lucros de U\$ 101 milhões por franquia.

Isso pode ser comprovado analisando o documentário feito por \citeonline{INVESTOPEDIA2019} no qual está descrito que, segundo a \textit{Bloomberg}\footnote{A \citeonline[online]{BLOOMBERG2019} oferece notícias, dados, análises e vídeos aos negócios e mercados ao mundo.\label{Bloomberg}}, estima-se um faturamento aproximado de U\$ 15 bilhões durante a temporada de 2018, acima das estimativas de U\$ 14,2 bilhões em 2017 e U\$ 13,3 bilhões em 2016.

% Isso pode ser comprovado analisando o documentário feito por \citeonline{INVESTOPEDIA2019}, no qual ele descreve que, segundo a \textit{Bloomberg} - A \citeonline[online]{BLOOMBERG2019} oferece notícias, dados, análises e vídeos aos negócios e mercados ao mundo -   estimou um faturamento aproximado de U\$ 15 bilhões durante a temporada de 2018, acima das estimativas de U\$ 14,2 bilhões em 2017 e U\$ 13,3 bilhões em 2016.

A \textit{NFL} mantêm seus fluxos de receita categorizados em \textit{“national revenue” and “local revenue”} (Receita nacional e local). A receita nacional está relacionada a acordos de TV, \textit{merchandising} e licenciamentos negociados nacionalmente pela própria \textit{NFL}.  Esse valor é dividido igualmente entre as equipes, independentemente do seu desempenho individual. O relatório anual de 2018 do time profissional de futebol americano \textit{Green Bay Packer} apresenta um ganho de aproximadamente U\$ 8,1 bilhões em receita nacional da \textit{NFL} (Aproximadamente  U\$ 255 milhões para cada uma das 32 equipes). Já a receita local consiste em vendas de ingressos, concessões e patrocinadores obtidos pelas próprias equipes. O \textit{Green Bay Packer} fechou o ano de 2018 com cerca de U\$ 196 milhões em receita local, o que representa cerca de 43\% da sua receita total nesse mesmo ano \cite{INVESTOPEDIA2019}.

Outra estatística importante é a apresentada por \citeonline{ESPN2019}, que relata um aumento significativo na audiência da rede de TV \textit{ESPN} durante a 53\textsuperscript{\underline{a}} edição do \textit{Super Bowl}, evento final da \textit{NFL}. O jornal \citeonline{FOLHASP2019} registrou um crescimento de 33\% na audiência da temporada de 2018 da \textit{NFL} no canal de esportes \textit{ESPN} Brasil, se comparado com a temporada anterior. “Fora os Estados Unidos, o Brasil é o segundo país mais interessado do mundo na \textit{NFL} atualmente, perdendo apenas para o México \cite{FOLHASP2019}.”