\chapter{\textbf{INTRODUÇÃO}}
\label{cap-introducao}
A evolução tecnológica expandiu consideravelmente nos dias atuais, proporcionando um avanço descomunal de vários hardwares poderosíssimos capazes de realizar tarefas jamais vistas pelos gênios antigos da computação. Devido a isso, imersões de várias técnicas vêm sendo estudadas progressivamente nos últimos anos, visto que atualmente existem equipamentos capazes de realizá-las em um curto prazo de tempo e de maneira mais eficaz.

A área de visão computacional se tornou de grande interesse nos tempos atuais devido à sua complexidade ao realizar processamentos de imagens e extrair o maior número de informações desta. O campo de processamento de imagem complementa esta parte, utilizando técnicas matemáticas e probabilísticas para corrigir os parâmetros visuais da imagem que será processada.

Por mais que existam alguns problemas nesta área como, por exemplo, a captura de imagens, processamento, alta precisão, similaridade, dentre outros, os estudos buscam aprimorar a ferramenta para que o campo de visão computacional se aproxime ou até ultrapasse, em algumas situações, à eficiência da uma visão humana.

O futebol americano é um esporte bastante popular nos Estados Unidos da América (EUA), na frente até mesmo de grandes esportes como, por exemplo, \textit{baseball}, \textit{basketball} e \textit{football}. Em 1869, nos EUA, aconteceu o primeiro jogo de futebol americano no mundo, onde a Universidade de \textit{Princeton} recebia o time da Universidade de \textit{Rugters}. Na época, o futebol americano ainda estava tomando forma, sem muitas regras e muito confuso. Mesmo assim, o esporte se popularizou, principalmente dentre os universitários \cite{RODRIGUES2014}. No Brasil o esporte tem se popularizado nos últimos anos devido ao fato das redes de TV por assinatura transmitirem o evento em larga escala (\autoref{mercado-do-futebol-americano}). 

\section{\textbf{{Objetivo geral}}}
\label{objetivo-geral}
Apresentar uma ferramenta que seja capaz de identificar um jogador de futebol americano dentro de campo. Para isso, a identificação será feita através de uma ferramenta a ser desenvolvida utilizando visão computacional e a biblioteca de processamento de imagens \textit{OpenCV}. 


\section{\textbf{{Objetivos específicos}}}
 \begin{itemize}
\item Classificar as imagens de um jogador de futebol americano e extrair o maior numero de informações;

\item Treinar o algoritmo seguindo as informações classificadas;

\item Elaborar um modelo de busca com o conjunto de informações processadas sobre jogadores de futebol americano;
   
\item Capturar, através de um dispositivo de entrada de vídeo, as imagens de uma partidade futebol americano;
   
\item Identificar o jogador seguindo o modelo de busca;
   
\item Analisar o percentual de acertos e erros da ferramenta.
   
 \end{itemize}
\section{\textbf{Justificativa}}

A área de visão computacional tem crescido de forma significativa no mundo presente, devido ao avanço tecnológico. Várias informações são capturadas e o volume de dados encontram-se crescendo progressivamente. Sendo assim, inumeras aplicações que utilizam a tecnologia de visão computacional estão sendo desenvolvidas para auxiliar diversas áreas, como por exemplo a medicina, segurança e esporte.

Este crescimento ocorre principalmente devido ao aumento da utilização de dispositivos móveis. Não é difícil se deparar com vários equipamentos que já utilizam essa tecnologia para alguma função, seja ela para melhorar algo, seja para realizar a identificação de algum objeto ou biometria de segurança, como por exemplo o \textit{Face ID} (Identidade de Rosto) da \textit{Apple} e o \textit{Google Lens}.

Sendo assim, a visão computacional pode auxiliar na questão relacionada ao reconhecimento de um jogador em uma partida de futebol americano.

As jogadas realizadas em futebol americano são de total contato físico e de alta velocidade. As transições dos jogadores são feitas em vários momentos para que as jogadas certas possam acontecer de acordo com a tática traçada pelo técnico. A leitura dessas substituições rápidas são feitas a olho humano, onde podem ocorrer equívocos e possíveis erros na identificação dos jogadores.

Outro fator crucial é que o mercado de futebol americano é muito valioso. Conforme descrito com mais detalhes na \autoref{mercado-do-futebol-americano} deste projeto, a \textit{NFL - National Football League} (Liga Nacional de Futebol) é a liga desportiva mais rica do mundo. A revista \citeonline{FORBES2019} descreve que a liga investe grandemente em ferramentas de \textit{marketing} digital e tecnologias que auxiliam no desempenho dos jogos. Dan Lovinger, vice-presidente executivo de vendas de publicidade da rede de transmissão \textit{NBC Sports}, informa que o preço médio de um comercial de 30 segundo no ar gira em torno de U\$ 5 milhões, totalizando um gasto médio de aproximadamente U\$ 500 milhões durante apenas uma partida de futebol americano. O show do intervalo patrocinado pela empresa \textit{Pepsi} mantêm um acordo que custa cerca de U\$ 7 milhões por ano, segundo o \textit{Sports Business Journal}. O valor dos ingressos do \textit{Super Bowl} varia entre U\$ 950 e U\$ 5 mil (Assentos \textit{Premium}), onde esses valores podem alcançar níveis mais altos em mercados paralelos. Outro fator que esta em grande crescimento dentre a população que acompanha a \textit{NFL} é o numero de apostas no evento \textit{Super Bowl}, que é denominado o melhor evento para realizar apostas. Na 51\textsuperscript{\underline{a}} edição do evento, o \textit{Nevada Gaming Control Board} (Conselho de Controle de Jogos de Nevada) registrou um recorde de U\$ 138,5 milhões em apostas \cite{FORBES2018}.

Com base nisso, pode-se reconhecer a importância de realizar um tratamento minucioso nos dados das partidas, para que não ocorra nenhum equívoco dentro de campo que pode comprometer algum índice.
\section{\textbf{Organização do trabalho}}

Este trabalho é composto por cinco capítulos, estruturados da seguinte forma: o \autoref{cap-fundamentos-conceituais} é composto pelos fundamentos conceituais que foram necessários para o entendimento da área de visão computacional. No \autoref{metodologia} foi abordada a metodologia utilizada para a construção deste trabalho.  Subsequente, o \autoref{desenvolvimento} relata todo o desenvolvimento do trabalho, ressaltando todas as etapas de funcionamento, descrição do sistema,  restrições, riscos e exclusões do projeto, requisitos funcionais e não-funcionais, explicação do código, etapa de testes e análise do \textit{software} e avaliação dos resultados. Por fim, o \autoref{consideracoes_finais} apresenta as considerações finais obtidas com o projeto, seguida das propostas de novos estudos.