\subsection{Tecnologias aplicadas no esporte}

O contexto da tecnologia atual vem sendo aplicado a passos largos na área desportiva. Tal ação proporciona uma enorme mudança nos resultados, gerando discussões relacionadas ao benefício e malefício desse progresso.

\citeonline{KATZ1989} explicam que com o uso de técnicas do meio desportivo incorporadas a tecnologia, é possível ampliar a performance e a inteligência dos atletas, fazendo com que as habilidades dos atletas seja cada vez mais aprimorada.

Isso ocorre porque atualmente os atletas possuem um grande número de tecnologias para os auxiliarem no esporte. Vários \textit{softwares} e ferramentas auxiliam os profissionais a estarem sempre preparados fisicamente e psicologicamente para exercer a sua tarefa.

A esgrima é um dos esportes que mais utiliza a tecnologia a seu favor, onde um colete feito de metal foi criado para emitir um pulso elétrico quando a espada o tocar. No \textit{taekwondo} também existe sensores de toque, que ficam acoplados abaixo do colete acolchoado que serve para reduzir o impacto físico no corpo do atleta. Com isso, a tecnologia proporcionou uma possibilidade de calcular até mesmo a potência de cada dano sofrido pelos atletas \cite{SCHATTENBERG2013}.

Além dos rádios comunicadores utilizados pelos árbitros se comuniquem dentro de campo, o futebol recebeu algumas tecnologias para que as partidas se tornem mais justas, como por exemplo os sensores implantando dentro das bolas. O dispositivo consegue verificar se a bola ultrapassou a linha delimitadora e, caso aconteça, o sensor emite imediatamente um alerta para os árbitros \cite{G12012}.

Outra tecnologia implementada recentemente no futebol é o \textit{VAR - Video Assistant Referee}, mais conhecida como árbitro de vídeo. Essa tecnologia surgiu com a ideia de minimizar ao máximo o erro humano dentro das partidas de futebol. No entanto, o \textit{VAR} está restrito à análise somente em situações de gol, pênalti e cartão vermelho. Sendo assim, além dos árbitros que já existiam dentro de campo, agora existe a equipe responsável pelo \textit{VAR}, que analisa os lances de uma forma mais criteriosa, usando os recursos tecnológicos que conseguem captar detalhadamente cada jogada dentro de campo \cite{CANALTECH2019}.

Apesar de tanta tecnologia, a reportagem feita por \citeonline{ESTADAO2019} apresenta um atraso de 46\% nas tomadas de decisões dos árbitros  nos lances com checagem do \textit{VAR}, o equivalente a 1:50 minutos.

Já o \textit{VAR} na \textit{NFL} se chama \textit{Instant Replay} (Replay Instantâneo). Na liga em questão, todos os jogos contam com a avaliação de imagem por parte da arbitragem. “Em 2017, apenas 429 marcações, dentre 39.967 jogadas, sofreram revisões. Desde 1999, 37\% das marcações em campo foram alteradas após avaliação". As avaliações levam em média 1:44 minutos para serem analisadas \cite{VARESPN2018}.
