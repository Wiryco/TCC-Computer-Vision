\section{\textbf{{Futebol Americano}}}
\label{futebol-americano}

O futebol americano é um esporte que se destaca devido ao porte físico de seus atletas e as disputas de força dentro de campo. O tempo de duração de uma partida pode demorar várias horas dependendo das pontuações em questão.

Do mesmo modo, vários jogadores participam de inúmeras jogadas dentro de campo, e é muito complexo acompanhar todas elas de forma detalhada, até mesmo para as pessoas que são treinadas e capacitadas para as analisarem.

O futebol americano consiste de uma série de jogadas, objetivando alcançar jardas\footnote{Jarda é um termo muito utilizado em países anglo-americanos que representam distâncias curtas. No futebol americano, as distâncias percorridas pelos jogadores são representadas por jardas. Cada jarda corresponde a aproximadamente 0,91 metros \cite{BRASILESCOLA2019}.\label{Jardas}} para somar mais pontos que seu adversário. Cada jogada consiste em 4 tentativas de avançar mais jardas do campo adversário. São 22 jogadores dentro de campo com possibilidades infinitas de substituição enquanto a bola não estiver em jogo. Um time tenta avançar e outro tenta impedir esse avanço: caso o time atacante consiga as jardas, ele continua com a posse de bola; caso contrário, o time adversário recebe a bola de volta e inicia a sua tentativa de ganhar jardas e chegar a \textit{End Zone} (Zona Final) \cite{NFL2019}.

Sua origem, datada em 1876, teve suas regras e seu modo de jogo evoluídas até chegar a um ponto onde a tecnologia pareava com o esporte, proporcionando uma velocidade acima da média nos resultados e na competitividade entre os jogadores da modalidade. Os treinadores começaram a utilizar alguns recursos para conseguir os melhores resultados dentro e fora de campo. Assim, a tecnologia se tornou um dos pilares de um time de ponta ao longo das competições. Sabe-se que o esporte é algo totalmente imprevisível mas existem pessoas que discordam dessa afirmação. De acordo com \citeonline{NEPOMUCENO2012}, todos os times que utilizam estatísticas e análise de dados têm vantagens sobre os seus adversários.

Atualmente, os dados obtidos manualmente por times e técnicos são coletados e em seguida, analisados por ferramentas de estatística. Dessa forma, se torna mais simples reconhecer os padrões e entregar informações que não são obtidas usando apenas a capacidade cognitiva e intelectual humana.

Com um número enorme de jogadores para avaliar durante a temporada do futebol profissional, o melhor modo para que treinadores, dirigentes e olheiros das equipes possam ter melhores hipóteses de quais jogadores possuem o potencial para agregar um melhor desempenho ao time, é agrupar candidatos em um lugar onde todos pudessem mostrar suas habilidades. O \textit{Draft} é um evento anual onde os times grandes podem analisar e selecionar jogadores de futebol americano universitário para reforçar o seu elenco. Esse é um dos maiores eventos que acontece na intertemporada da \textit{NFL}. Já o \textit{Combine} é um evento que acontece antes do \textit{Draft}, que proporciona aos executivos, técnicos e responsáveis pelo departamento pessoal, realizarem uma análise da capacidade física e mental dos jogadores universitários \cite{MCGEE2003}.

Sendo assim, ter dados precisos sobre as habilidades destes jogadores e compará-las com outros jogadores participantes do \textit{Combine} e do próprio elenco também é útil para auxílio nas estratégias e no posicionamento de cada atleta selecionado.

De acordo com \citeonline{BASS2012} um exemplo claro disto é a avaliação dos \textit{quarterbacks}. Com a grande necessidade dos times por um jogador que lidere os ataques, observar bem os jogadores da posição durante o \textit{Combine} pode ser decisivo no momento em que um time define se vale a pena escolher um \textit{quarterback} no \textit{Draft} (Evento que acontece depois do \textit{Combine}), manter um que já está no elenco ou partir para o mercado em busca do substituto que possa cumprir com as expectativas da equipe.

Para os jogadores, o \textit{Combine} é a melhor forma de tentar impressionar os observadores das equipes e convencê-los de que ele pode ser o atleta ideal. Quem se destaca com bons números durante os testes pode ver sua cotação aumentar nas listas das equipes e até ter a chance de ser escolhido durante a primeira rodada do \textit{Draft}, uma oportunidade que apenas 32 jogadores conseguem a cada ano.

\subsection{Tecnologias aplicadas no esporte}

O contexto da tecnologia atual vem sendo aplicado a passos largos na área desportiva. Tal ação proporciona uma enorme mudança nos resultados, gerando discussões relacionadas ao benefício e malefício desse progresso.

\citeonline{KATZ1989} explicam que com o uso de técnicas do meio desportivo incorporadas a tecnologia, é possível ampliar a performance e a inteligência dos atletas, fazendo com que as habilidades dos atletas seja cada vez mais aprimorada.

Isso ocorre porque atualmente os atletas possuem um grande número de tecnologias para os auxiliarem no esporte. Vários \textit{softwares} e ferramentas auxiliam os profissionais a estarem sempre preparados fisicamente e psicologicamente para exercer a sua tarefa.

A esgrima é um dos esportes que mais utiliza a tecnologia a seu favor, onde um colete feito de metal foi criado para emitir um pulso elétrico quando a espada o tocar. No \textit{taekwondo} também existe sensores de toque, que ficam acoplados abaixo do colete acolchoado que serve para reduzir o impacto físico no corpo do atleta. Com isso, a tecnologia proporcionou uma possibilidade de calcular até mesmo a potência de cada dano sofrido pelos atletas \cite{SCHATTENBERG2013}.

Além dos rádios comunicadores utilizados pelos árbitros se comuniquem dentro de campo, o futebol recebeu algumas tecnologias para que as partidas se tornem mais justas, como por exemplo os sensores implantando dentro das bolas. O dispositivo consegue verificar se a bola ultrapassou a linha delimitadora e, caso aconteça, o sensor emite imediatamente um alerta para os árbitros \cite{G12012}.

Outra tecnologia implementada recentemente no futebol é o \textit{VAR - Video Assistant Refere}, mais conhecida como árbitro de vídeo. Essa tecnologia surgiu com a ideia de minimizar ao máximo o erro humano dentro das partidas de futebol. No entanto, o \textit{VAR} está restrito à análise somente em situações de gol, pênalti e cartão vermelho. Sendo assim, além dos árbitros que já existiam dentro de campo, agora existe a equipe responsável pelo \textit{VAR}, que analisa os lances de uma forma mais criteriosa, usando os recursos tecnológicos que conseguem captar detalhadamente cada jogada dentro de campo \cite{CANALTECH2019}.

Apesar de tanta tecnologia, a reportagem feita por \citeonline{ESTADAO2019} apresenta um atraso de 46\% nas tomadas de decisões dos árbitros  nos lances com checagem do \textit{VAR}, o equivalente a 1:50 minutos.

Já o \textit{VAR} na \textit{NFL} se chama \textit{Instant Replay} (Replay Instantâneo). Na liga em questão, todos os jogos contam com a avaliação de imagem por parte da arbitragem. “Em 2017, apenas 429 marcações, dentre 39.967 jogadas, sofreram revisões. Desde 1999, 37\% das marcações em campo foram alteradas após avaliação". As avaliações levam em média 1:44 minutos para serem analisadas \cite{VARESPN2018}.


\subsection{O mercado do futebol americano}
\label{mercado-do-futebol-americano}

O futebol americano é o esporte mais popular nos Estados Unidos e, segundo \citeonline{FORBES2018}, a \textit{NFL} é a liga esportiva mais rica do mundo, girando cerca de U\$ 2,5 bilhões por cada time participante, operando com lucros de U\$ 101 milhões por franquia.

Isso pode ser comprovado analisando o documentário feito por \citeonline{INVESTOPEDIA2019} no qual está descrito que, segundo a \textit{Bloomberg}\footnote{A \citeonline[online]{BLOOMBERG2019} oferece notícias, dados, análises e vídeos aos negócios e mercados ao mundo.\label{Bloomberg}}, estima-se um faturamento aproximado de U\$ 15 bilhões durante a temporada de 2018, acima das estimativas de U\$ 14,2 bilhões em 2017 e U\$ 13,3 bilhões em 2016.

% Isso pode ser comprovado analisando o documentário feito por \citeonline{INVESTOPEDIA2019}, no qual ele descreve que, segundo a \textit{Bloomberg} - A \citeonline[online]{BLOOMBERG2019} oferece notícias, dados, análises e vídeos aos negócios e mercados ao mundo -   estimou um faturamento aproximado de U\$ 15 bilhões durante a temporada de 2018, acima das estimativas de U\$ 14,2 bilhões em 2017 e U\$ 13,3 bilhões em 2016.

A \textit{NFL} mantêm seus fluxos de receita categorizados em \textit{“national revenue” and “local revenue”} (Receita nacional e local). A receita nacional esta relacionada a acordos de TV, \textit{merchandising} e licenciamentos negociados nacionalmente pela própria \textit{NFL}.  Esse valor é dividido igualmente entre as equipes, independentemente do seu desempenho individual. O relatório anual de 2018 do time profissional de futebol americano \textit{Green Bay Packer} apresenta um ganho de aproximadamente U\$ 8,1 bilhões em receita nacional da \textit{NFL} (Aproximadamente  U\$ 255 milhões para cada uma das 32 equipes). Já a receita local consiste em vendas de ingressos, concessões e patrocinadores obtidos pelas próprias equipes. O \textit{Green Bay Packer} fechou o ano de 2018 com cerca de U\$ 196 milhões em receita local, o que representa cerca de 43\% da sua receita total nesse mesmo ano \cite{INVESTOPEDIA2019}.

Outra estatística importante é a apresentada por \citeonline{ESPN2019}, que relata um aumento significativo na audiência da rede de TV \textit{ESPN} durante a 53\textsuperscript{\underline{a}} edição do \textit{Super Bowl}, evento final da \textit{NFL}. O jornal \citeonline{FOLHASP2019} registrou um crescimento de 33\% na audiência da temporada de 2018 da \textit{NFL} no canal de esportes \textit{ESPN} Brasil, se comparado com a temporada anterior. “Fora os Estados Unidos, o Brasil é o segundo país mais interessado do mundo na \textit{NFL} atualmente, perdendo apenas para o México \cite{FOLHASP2019}.”