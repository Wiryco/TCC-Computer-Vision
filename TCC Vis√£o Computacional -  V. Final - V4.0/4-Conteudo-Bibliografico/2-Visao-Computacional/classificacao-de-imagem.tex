\subsubsection{\textit{Classificação de imagens}}
\label{classificacao-de-imagem}
% \textbf coloca o subsection em negrito

A classificação de imagem é a última etapa do processamento de imagem (\autoref{fig_etapas-processamento-imagem}). Em síntese, esta etapa é responsável por realizar a classificação das imagens levando em consideração as suas características.

Entretanto, segundo \citeonline{LIBERMAN97}, nessa etapa do processamento, o grau de abstração de cada característica da imagem pode ser classificado em três níveis distintos: baixo, médio e alto.

No processo de baixo nível são utilizados os \textit{pixels} originais da imagem como parâmetros de comparação, para que no final do processo sejam gerados propriedades da imagem, em forma de valores numéricos, associadas a cada \textit{pixel} que foi analisado. Sequencialmente, o nível médio coleta essas propriedades numéricas geradas pelo processo de baixo nível e produz uma lista de características da imagem. Por fim, o processo de alto nível reúne estas características ocasionadas pelo processo anterior buscando interpretá-las, formando assim o conteúdo da imagem.

Segundo \citeonline{LIBERMAN97}, o processo de classificação ou interpretação de uma imagem é a parte mais inteligente da visão computacional. O autor do artigo cita que essa é uma das etapas de maior nível, no qual permite-se obter a “compreensão e a descrição final do fenômeno inicial”.

Para complementar, \citeonline{LIBERMAN97} explica que o processo de classificação de imagem possui duas técnicas para realizar suas tarefas, sendo divididas em supervisionada ou não-supervisionada. A classificação não-supervisionada consiste em um agrupamento automático de sequências similares de uma imagem analisada. Conforme já prescrito neste trabalho no contexto de segmentação interativa e agora completado por \citeonline{LIBERMAN97}, nessa etapa a imagem será segmentada em um número indeterminado de classes, no qual o usuário também será responsável por gerenciar essas classes a fim de alcançar seus objetivos.

De acordo com \citeonline{MAXIMO2005}, no processo de classificação supervisionada, o analista ou usuário filtra as classes de informações seguindo os seus padrões de interesse e separa, na imagem, as regiões que satisfazem essas classes. Após a delimitação das classes, a técnica analisará as mesmas com o objetivo de delimitar \textit{pixels} que serão utilizados como parâmetros para a busca de demais \textit{pixels}.

Simplificadamente, a técnica de classificação supervisionada utiliza amostras de características coletadas durante o processo para identificar cada \textit{pixel} definido como \textit{pixel} desconhecido, ou seja, tons de \textit{pixels} que não fazem parte das características já coletadas anteriormente seguindo os filtros definidos pelo usuário.

\begin{itemize}
\raggedright \item \textit{Haar Cascade}
\end{itemize}
 
Para complementar o assunto citado acima, a técnica de \textit{Haar Cascade} utiliza a classificação de imagens para obter um padrão de características que foram extraídas da imagem. Essa classificação é utilizada para montar uma cascata de características, ou seja, um conjunto de imagens. A principal base para a detecção de objetos do classificador \textit{Haar} são os recursos extraídos da imagem, ou seja, ao invés de usar os valores de intensidade de um \textit{pixel}, usa-se as alterações nos valores de contraste entre os grupos retangulares dos \textit{pixels}. Basicamente, \textit{Haar Cascade} é baseada em \textit{Haar Wavelets}, que utiliza uma sequência de funções redimensionadas em quadrantes que juntas formam uma base de \textit{wavelets} \cite{WILSON2006}.

A detecção de objetos e faces utilizando técnicas de classificadores em cascata baseados em recursos \textit{Haar} é um método eficaz proposto por \citeonline{VIOLA2001} em seu artigo. Á abordagem é baseada em \textit{machine learning} (aprendizado de máquina) no qual a função cascata é treinada a partir de enumeras imagens positivas e negativas. Através desse recurso, pode-se obter a eficiência em detecção de objetos em outras imagens \cite{OpenCV}. Este trabalho não descreve os detalhes do funcionamento do detector de Viola-Jones. O leitor interessado pode encontrá-lo em \cite{VIOLA2001}.

\begin{itemize}
\raggedright \item \label{itm:machinelearning} \textit{Machine Learning}
\end{itemize}

O \textit{haar cascade} pode ser descrito com uma técnica utilizada para gerar um conjunto com todas as características da imagem.  Portanto, é possível utilizar o \textit{machine learning} para obter melhores resultados, porque ele utiliza como parâmetro de entrada o conjunto de características para ajustas e adaptar todos os pontos de interesse do conjunto.  Após a análise do algoritmo, o reconhecimento de imagem pode ser aprimorado. O reconhecimento de padrões e o \textit{machine learning} são técnicas que passaram por um crescimento substancial ao longo dos tempos \cite{BISHOP2006}.
 
Sendo assim, o \textit{machine learning} pode ser definido como a área que estuda alternativas para fazer as máquinas executarem tarefas que se aproximam das atividades humanas. Essa tecnologia possibilitou a criação de sistemas capazes de realizar várias atividades de forma automatizada, ou seja, a tecnologia pode ser programada para realizar tomada de decisões a partir dos dados que são utilizados para treinar o algoritmo \cite{MONARD2003}.

Devido ao crescimento exponencial do aprendizado de máquina, diversos sistemas foram criados para resolver problemas. Isso não significa que existe todos os modelos já feitos para resolver todo o tipo de problema existente. Sendo assim, é necessário entender a capacidade e as limitações da tecnologia para analisar a sua usabilidade, pois seu uso requer muito poder de processamento.