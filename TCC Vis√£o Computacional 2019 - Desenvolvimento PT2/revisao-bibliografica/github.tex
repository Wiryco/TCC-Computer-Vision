\subsubsection{\textit{GitHub}}


Já o \textit{GitHub} é uma plataforma de hospedagem de códigos que utiliza o \textit{Git} como controle de versão. O \textit{GitHub} possui uma grande interação com repositórios \textit{Git}, concentrando uma grande comunidade de desenvolvedores que colaboram para milhões de projetos \cite{CHACON2014}.

Ainda segundo \citeonline{CHACON2014}, o \textit{GitHub} hospeda a maior porcentagem de repositórios \textit{Git}. Isso ocorre porque a plataforma também disponibiliza recursos de gerenciamento de código, como por exemplo o rastreamento de problemas, revisão de código, edição \textit{online} do código, dentre outros.