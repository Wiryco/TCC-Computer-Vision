\section{\textbf{Etapa de aprendizado}}
\label{explicacao-aprendizado}

\newcommand{\refanexo}[1]{\hyperref[#1]{Anexo~\ref{#1}}}

A etapa de aprendizado consiste basicamente em abordar quais imagens foram utilizadas como parâmetros de entrada para o treinamento do algoritmo. Essa etapa se faz necessária para entender quais características de jogadores de futebol americano foram analisadas para criar o melhor modelo de busca, que foi utilizado para realizar as comparações por similaridade.

Para isso, foi utilizado o classificador \textit{haar cascade} para realizar a extração de características de todas as imagens dos jogadores. Essa extração de características foi feita utilizando o método não-supervisionado de extração, ou seja, o algoritmo fica responsável por analisar as imagens e ele mesmo define quais são os pontos de interesse contidos nela. A \autoref{classificacao-de-imagem} explica detalhadamente como essa técnica funciona.

As imagens positivas, ou seja, imagens de jogadores de futebol americano, foram escolhidas seguindo as situações reais dentro de campo, ou seja, jogadores em movimento; jogadores em disputa de bola; jogadores parados, dentre outras situações. Portanto, foi feito um diretório contendo todas as imagens positivas, que serão utilizadas como parâmetro de entrada para que o algoritmo possa analisa e extrair todas as suas características. O diretório de imagens positivas pode ser visualizado no \refanexo{anexoA}.

Já o \refanexo{anexoB} representa todas as imagens negativas, ou seja, imagens que não contém jogadores de futebol americano.

Sendo assim, o algoritmo por si só fica responsável por ler todas essas imagens, positivas e negativas, extraindo todas as suas características. Simultaneamente, o \textit{machine learning} fica responsável por aprender o que é e o que não é um jogador de futebol americano. Em seguida, pode-se obter um modelo de busca mais robusto que contém o melhor padrão de característica para reconhecimento dos jogadores dentro de campo.